\documentclass[14pt]{article}
%\usepackage[cp866nav]{inputenc}
\usepackage[cp1251]{inputenc}
\usepackage[intlimits,sumlimits]{amsmath}
\usepackage{enumerate,graphicx,dcolumn,amsthm}
\usepackage[english,russian,ukrainian]{babel}

\usepackage{floatflt}

\usepackage{tabularx}
\renewcommand{\tabularxcolumn}[1]{m{#1}}
\usepackage{hhline}
\usepackage{multirow}

 \usepackage[left=50mm,right=60mm,top=20mm,bottom=16mm,bindingoffset=0mm,footskip=6mm,includefoot]{geometry}
\setlength{\parindent}{5mm}
\usepackage[hang,flushmargin]{footmisc}




\numberwithin{equation}{part}



\begin{document}           % End of preamble and beginning of text.
\fontsize{14}{16pt}\selectfont

\noindent
Auxiliary  function (Cibils  and  Buitrago method)
\begin{equation*}
F_a(V)=V-V_a\ln I,
\end{equation*}
$V_a$ is an arbitrary voltage, $V_a\geq99.5I_sR_s+nkT/q$,

\noindent
$I_s$ is the saturation current,

\noindent
$R_s$ is the series resistance,

\noindent
$n$ is the ideality factor.

If $I_{min,a}$ is the current value at the voltage $V_{min}$ where the function $F_a(V)$ exhibits a minimum, than
the plot of $I_{min,a}$ vs $V_a$ is expected to be linear:
\begin{equation*}
I_{min,a}=V_a/R_s-nkT/qR_s\,.
\end{equation*}

ffffffffffffffffffffffffffffffffffff
\vspace{5mm}

Auxiliary  function (Cheung and Cheung method)
\begin{equation*}
C(I)=dV/d(\ln I).
\end{equation*}
If $V_d>3kT/q$ then this plot must be linear  functions of the  current
\begin{equation*}
C(I)=R_sI+nkT/q
\end{equation*}
$R_s$ is the series resistance,

\noindent
$n$ is the ideality factor.

\vspace{5mm}
aaaaaaaaaaaaaaaaaaaaaaaaaaaaaa
\vspace{5mm}

Forward branch is approximated by
\begin{equation*}
\frac{I}{1-\exp(-qV/kT)}=I_s\exp\left(\frac{qV}{nkT}\right)
\end{equation*}
\noindent
$n$ is the ideality factor,

\noindent
$I_s$ s the saturation current
\begin{equation*}
I_s=AA^*\,T^2\exp\left(-\frac{q\Phi_b}{kT}\right)
\end{equation*}
$A$ is the diode area,

\noindent
$A^*$ is the effective Richardson constant,

\noindent
$\Phi_b$ is zero bias Schottky barrier height.

\pagebreak

Reverse branch is approximated by
\begin{equation*}
\frac{I}{1-\exp(-qV/kT)}=I_s\exp\left(\frac{qV}{nkT}\right)
\end{equation*}
\noindent
$n$ is the ideality factor,

\noindent
$I_s$ is the saturation current
\begin{equation*}
I_s=AA^*\,T^2\exp\left(-\frac{q\Phi_b}{kT}\right)
\end{equation*}
$A$ is the diode area,

\noindent
$A^*$ is the effective Richardson constant,

\noindent
$\Phi_b$ is zero bias Schottky barrier height.

\vspace{5mm}
aaaaaaaaaaaaaaaaaaaaaaaaaaaaaa
\vspace{5mm}

\noindent
Werner method based on the plot of $Y$ vs $X$, where
\begin{equation*}
X=\frac{dI}{dV};\qquad Y=\frac{(dI/dV)}{I}
\end{equation*}
If $(V-IR_s)\gg nkT/q$, Werner's function approximation is:
\begin{equation*}
Y=\frac{q}{nkT}(1-R_sX)
\end{equation*}
$R_s$ is the series resistance,

\noindent
$n$ is the ideality factor.

\vspace{5mm}
aaaaaaaaaaaaaaaaaaaaaaaaaaaaaa
\vspace{5mm}

Auxiliary  function is
\begin{equation*}
H(I)=V-\frac{nkT}{q}\ln\left(\frac{I}{AA^*T^2}\right).
\end{equation*}
$n$ is the ideality factor,

\noindent
$A$ is the diode area,

\noindent
$A^*$ is the effective Richardson constant.

\noindent If $(V-IR_s)>3kT/q$ then this plot must be linear  functions of the  current
\begin{equation*}
H(I)=n\Phi_b+IR_s
\end{equation*}
$R_s$ is the series resistance,

\noindent
$\Phi_b$ is zero bias Schottky barrier height.


\vspace{5mm}
aaaaaaaaaaaaaaaaaaaaaaaaaaaaaa
\vspace{5mm}

Mikhelashvili's auxiliary functions are:
\begin{eqnarray*}
\alpha(V)&=&d(\ln I)/d(\ln V),\\
\beta(V)&=&d(\ln \alpha)/d(\ln V),\\
n(V)&=&qV(\alpha-1+\beta)/\alpha^{\,2} kT,\\
R_s(V)&=&V(1-\beta)/I\alpha^{\,2},
\end{eqnarray*}
$n$ is the ideality factor, $R_s$ is the series resistance.

If $\alpha_{m}$ and $V_{m}$ are the coordinates of maximum point in the plot $\alpha$ vs $V$,
$I_{m}$ is the current value at the voltage $V_{m}$ then
\begin{eqnarray*}
n&=&qV_m[\alpha_{m}-1]/kT\alpha_{m}^{\,2},\\
R_s&=&V_m/I_m\alpha_{m}^{\,2},\\
I_s&=&I_m\exp(-\alpha_{m}-1),\\
\Phi_b&=&kT\{\alpha_{m}+1-\ln[I_m/(AA^*T^2)]\},
\end{eqnarray*}
$\Phi_b$ is zero bias Schottky barrier height,
$I_s$ is the saturation current,
$A$ is the diode area,
$A^*$ is the effective Richardson constant.

\vspace{5mm}
aaaaaaaaaaaaaaaaaaaaaaaaaaaaaa
\vspace{5mm}

Method uses the functions array $\{F(I)\}$, where
\begin{equation*}
F(I)=V(I)-V_a\ln I,
\end{equation*}
$V_a$ is an arbitrary voltage.
Each $F(I)$ function is approximated by
\begin{equation*}
y(I)=c_1+c_2I+c_3\ln I
\end{equation*}
and parameters $c_1$, $c_2$ and $c_3$ is determined.
For $V>3kT/q$, the dependence of $I_a=-c_3/c_2$ on $V_a$ is expected to be linear:
\begin{equation*}
I_a(V_a)=V_a/R_s-nkT/qR_s.
\end{equation*}
$R_s$ is the series resistance,
$n$ is the ideality factor.

The barrier height $\Phi_b$ can be calculated
\begin{equation*}
\Phi_b=c_3/n+(kT/q)\cdot\ln\left(AA^*T^2\right).
\end{equation*}
$A$ is the diode area,
$A^*$ is the Richardson constant.

\pagebreak

The plot of $V$ vs $I$ is fitted by
\begin{equation*}
y(I)=c_1+c_2I+c_3\ln I
\end{equation*}
and parameters $c_1$, $c_2$ and $c_3$ is determined.
\begin{eqnarray*}
\label{eqGr1}
R_s&=&c_2\,,
\\
n&=&qc_3/kT\,,
\\
\Phi_b&=&kT/q\cdot\left[c_1/c_3+\ln\left(AA^*T^2\right)\right]\,,
\end{eqnarray*}
$R_s$ is the series resistance,
$n$ is the ideality factor,
$\Phi_b$ is zero bias Schottky barrier height,
$I_s$ is the saturation current,
$A$ is the diode area,
$A^*$ is the effective Richardson constant.

\vspace{5mm}
aaaaaaaaaaaaaaaaaaaaaaaaaaaaaa
\vspace{5mm}

Auxiliary  function is
\begin{equation*}
F(I)=\frac{V(I)}{2}-\frac{kT}{q}\ln\left(\frac{I}{AA^*T^2}\right).
\end{equation*}
The function is fitted by
\begin{equation*}
y(I)=c_1+c_2I+c_3\ln I
\end{equation*}
and parameters $c_1$, $c_2$ and $c_3$ is determined.
\begin{eqnarray*}
R_s&=&2c_2\,,
\\
n&=&\frac{2qc_3}{kT}+2\,,
\\
\Phi_b&=&\frac{2c_1}{n}+\frac{(2-n)kT}{nq}\ln\left(AA^*T^2\right)\,.
\end{eqnarray*}
$R_s$ is the series resistance,
$n$ is the ideality factor,
$\Phi_b$ is zero bias Schottky barrier height,
$A$ is the diode area,
$A^*$ is the effective Richardson constant.

\pagebreak

The coordinates of $j-$th point of auxiliary plot are calculated by
\begin{equation*}
Y_j=\frac{1}{I_j-I_1}\int_{V_1}^{V_j}I\,dV \quad\text{and}\quad X_j=\frac{I_j+I_1}{2},
\end{equation*}
$V_i$ and $I_i$ are the coordinates of $i-$th point of the $I$--$V$ curve,
$i\in(1,\ldots, N_p)$,
$j\in(2,\ldots, N_p)$.
The total number of points is  $(N_p-1)$.

The plot of $Y$ vs $X$ is expected to be linear:
\begin{equation*}
Y=nkT/q+R_sX,
\end{equation*}
$R_s$ is the series resistance,

\noindent
$n$ is the ideality factor.


\vspace{5mm}
aaaaaaaaaaaaaaaaaaaaaaaaaaaaaa
\vspace{5mm}

The coordinates of $k-$th point of auxiliary plot are calculated by
\begin{equation*}
Y_k=\frac{\ln(I_j/I_i)}{I_j-I_i} \quad\text{and}\quad X_k=\frac{V_j-V_i}{I_j-I_i},
\end{equation*}
$V_l$ and $I_l$ are the coordinates of $l-$th point of the $I$--$V$ curve,
$l\in(1,\ldots, N_p)$,
$i\in(1,\ldots, N_p-1)$,
$j\in(i+1,\ldots, N_p)$,
$k\in(1,\ldots, N_p(N_p-1)/2)$.

The plot of $Y$ vs $X$ is expected to be linear:
\begin{equation*}
Y=q(-R_s+X)/nkT.
\end{equation*}
$R_s$ is the series resistance,

\noindent
$n$ is the ideality factor.

\pagebreak

\begin{center}
Density of interface states $D_{is}$


   vs


energy of interface states with respect to the bottom of the conduction band $(E_c-E_{is})$
\end{center}
\vspace{-8mm}
\begin{eqnarray*}
D_{is}=\frac{\varepsilon_i\varepsilon_0(n-1)}{qd}\,,
\\
(E_c-E_{is})=(\Phi_b-V/n)\,,
\end{eqnarray*}
$\varepsilon_0$ and $\varepsilon_i$ are the permittivities of free space and insulator layer respectively,
$d$ is the thickness of the interfacial insulator layer,
$n$ is the ideality factor,
$\Phi_b$ is zero bias Schottky barrier height.
\begin{equation*}
[\,D_{is}\,]=\text{eV}^{-1}\text{cm}^{-2}\qquad [\,E_c-E_{is}\,]=\text{eV}\,.
\end{equation*}



\vspace{5mm}
aaaaaaaaaaaaaaaaaaaaaaaaaaaaaa
\vspace{5mm}

Approximation by
\begin{equation*}
I=I_s\exp\left(\frac{qV}{nkT}\right)\,,
\end{equation*}
$n$ is the ideality factor,

\noindent
$I_s$ is the saturation current,
\begin{equation*}
I_s=AA^*T^2\exp\left(-\frac{\Phi_b}{kT}\right)\,,
\end{equation*}
$A$ is the diode area,

\noindent
$A^*$ is the effective Richardson constant,

\noindent
$\Phi_b$ is zero bias Schottky barrier height.


\pagebreak

Auxiliary  function:
\begin{equation*}
F(V)=\frac{V}{\gamma}-\frac{kT}{q}\ln\left(\frac{I(V)}{AA^*T^2}\right),
\end{equation*}
$A$ is the diode area,
$A^*$ is the effective Richardson constant,
$\gamma$ is an arbitrary constant greater than the ideality factor $n$.

If $F(V_{min})$ and $V_{min}$ are the coordinates of minimum point in the plot $F(V)$ vs $V$ and
$I_{min}$ is the current value at the voltage $V_{min}$, then
\begin{eqnarray*}
\Phi_b&=&F(V_{min})+\frac{\gamma-n}{n}\left(\frac{V_{min}}{\gamma}-\frac{kT}{q}\right),
\\
R_s&=&\frac{(\gamma-n)kT}{qI_{min}}\,,
\end{eqnarray*}
$\Phi_b$ is zero bias Schottky barrier height,

\noindent
$R_s$ is the series resistance.


\vspace{5mm}
aaaaaaaaaaaaaaaaaaaaaaaaaaaaaa
\vspace{5mm}




\end{document}
